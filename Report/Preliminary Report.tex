\documentclass[acmsmall,screen,nonacm,review]{acmart}
\usepackage{tcolorbox}
\usepackage{minted}
\usepackage{hyperref}

\begin{comment}
  TODO: 
  -reduce pages ?
  -code macro
  -acknowledgements
  -adjust layout

  -minted settings :xleftmargin = 0.3cm
\end{comment}

\definecolor{bg}{rgb}{0.97,0.97,0.97}

\author{}
\email{}



\begin{document}


\title{An Introduction to HLSL}

\begin{abstract}

This document is an introduction to the shading language HLSL made for the course ``Seminar Modern Programming Languages'' at Ruhr-University Bochum 2025/2026.
It starts with a brief look at the history of HLSL, which explains the purpose of the language.
Following this are explanations of HLSL concepts including data types, semantics, flow control, intrinsics and shader models.
The last part is about the graphics pipeline. The pipeline stages are explained by looking at an animated cube and
observing the changes made by each stage. Each stage includes a brief explanation of the stage, a picture of the cube and 
the main function of each shader.

\noindent It is possible, that some concepts mentionend are not further explained or details are omitted, because they would go beyond the scope of this course. 
A list of concepts that are not explained can be found at the end of this document.

\end{abstract}

\maketitle

\section{History of HLSL}
When gpu's were introduced, each gpu required dedicated code. In 2002\cite{Direct3DGettingStarted}, Microsoft released DirectX 9.0 and with it the High Level Shader Language\cite{DirectX2003}.\@
The goal was to provide a flexible and developer-friendly environment that abstracts the different gpu implementations\cite{DirectX2003}.
HLSL is based on the c programming language\cite{DirectX2003} and adds additional concepts designed for gpu programming. 
The language implements multiple versions of shader models, each being an extension of the previous version\cite{HLSLShaderModel}.
The latest shader model is the 6th version, which is used by Direct3D 12. 



\section{Data Types}
HLSL supports various data types. This section gives an overview over the different Data Types.

\subsection{Scalar}
Scalar variables use a single component of a register. There are different kinds of scalar data types.
The main data types are:
\begin{itemize}
\item\textbf{bool}: true or false
\item\textbf{int}: 32-bit signed integer
\item\textbf{uint}: 32-bit unsigned integer
\item\textbf{half}: 16-bit floating point
\item\textbf{float}: 32-bit floating point
\item\textbf{double}: 64-bit floating point
\item\textbf{snorm float}: 32-bit signed-normalized float in the intervall [-1,1]
\item\textbf{unorm float}: 32-bit unsigned-normalized float in the intervall [0,1] 
\end{itemize}

\subsection{Vector}
Vectors are variables that use up to 4 components\cite{HLSLVector}. The type can be any of the scalar data types. A vector can be declared in two ways:

Declaration with a template: \colorbox{bg}{\textbf{vector}<Type, Components = [1,4]>}

\textit{Example:}
\begin{minted}[xleftmargin=0.3cm,bgcolor=bg]{hlsl}
//A four component float
vector<float, 4> ColorRGBA;

//A single component float, the same as a scalar float
vector<float, 1> RotationAngle; 
\end{minted}

Declaration as a type: \colorbox{bg}{\textbf{float[n]}, where n is a number in the intervall [1,4].}

\textit{Examples:}
\begin{minted}[xleftmargin=0.3cm,bgcolor=bg]{hlsl}
float4 vPosition :SV_Position;
float4 Color : COLOR;
float3 Normal : NORMAL;
\end{minted}

\noindent Accessing a component can be done by using the structure operator ``\textbf{.}''\cite{HLSLComponentMath} followd by the letters xyzw or rgba.

\noindent\textit{Examples:}
\begin{minted}[xleftmargin=0.3cm,bgcolor=bg]{hlsl}
Vector2.xy  //acceses the first and second component
Vector3.rgb //acceses the first second and third component
Vector4.yzw //accesses the second, third and fourth component
\end{minted}

\noindent The letters \textit{xyzw} or \textit{rgba} correspond to the components of a Vector.
\begin{itemize}
  \item\textit{x} and \textit{r} correspond to the first component.
  \item\textit{y} and \textit{g} correspond to the second component.
  \item\textit{z} and \textit{b} correspond to the third component.
  \item\textit{w} and \textit{a} correspond to the fourth component.
\end{itemize}




\subsection{Matrix}
Matrices are two dimensional grids made of scalar components\cite{HLSLMatrix}. HLSL allows matrices with one to four columns and rows respectively.

\noindent Like Vectors, there are two ways of declaring matrices:

Template Declaration:\colorbox{bg}{\textbf{matrix}<Type, Rows = [1,4], Cols = [1,4]>}

Type Declaration:\colorbox{bg}{Type[Rows]x[Columns]}

\noindent \textit{Examples:}

\begin{minted}[xleftmargin=0.3cm,bgcolor=bg]{hlsl}
//A 4 by 4 Matrix
float4x4 RollRotationMatrix = {
1.0f,0.0f,0.0f,0.0f,
0.0f,cos(RadAngle),-sin(RadAngle),0.0f,
0.0f,sin(RadAngle),cos(RadAngle),0.0f,
0.0f,0.0f,0.0f,1.0f};

//A 3 by 2 matrix, initialized with 0
int3x2 Matrix = {}; 

//A 3 by 3 uint matrix
matrix<uint,3,3> Mat3x3 
\end{minted}

\noindent Accessing a component of a matrix can be done multiple ways using the structure operator \textbf{.}\cite{HLSLComponentMath}. One way to index a component is \_mRC, where R and C correspond to Row/Column indices\cite{HLSLComponentMath}.

\noindent \textit{Example:} 

\begin{minted}[xleftmargin=0.3cm,bgcolor=bg]{hlsl}
//Accesses the first row of the RollRotationMatrix
float4 FirstRow = RollRotationMatrix._m00_m01_m02_m03; 
\end{minted}

\subsection{Buffers}

Each stage of the graphics pipeline has the possibilty to hold references to buffers provided by the cpu. To acces these buffers, HLSL provides multiple buffer types.

\noindent Examples of Buffer types:
\begin{itemize}
\item\textbf{cbuffer}:  A Constant Buffer Object
\item\textbf{RWBuffer}:  A Read/write Buffer Object
\item\textbf{RWTexture2D}: A 2d Texture Buffer Object
\end{itemize}


\textit{Buffer Examples:}

\begin{minted}[xleftmargin=0.3cm,bgcolor=bg]{hlsl}
// A Read/Write Texture2D Buffer Object 
RWTexture2D <float4> TextureCopy;

//A Constant Buffer Object
cbuffer CBuffer{                          
  float RotationAngle : packoffset(c0.x);
  float Width  : packoffset(c0.y);
  float Height : packoffset(c0.z);
  float ColorR : packoffset(c1.x);
  float ColorG : packoffset(c1.y);
  float ColorB : packoffset(c2.x);
  float ColorA : packoffset(c2.y);
};
\end{minted}

\noindent Each buffer object has its own way of accessing the data.

\noindent \textit{Examples:}

Variables in a constant buffer can be used as global variables.

\begin{minted}[xleftmargin=0.3cm,bgcolor=bg]{hlsl}
//Accessing the CBuffer Object from the previous example.
RotationAngle +=1;
\end{minted}

RWTexture2D data can be accesed like an array.
\begin{minted}[xleftmargin=0.3cm,bgcolor=bg]{hlsl}
//Accessing the TextureCopy from the previous example
float4 Color = TextureCopy[0];
\end{minted}

\subsection{Struct}
A Struct is a collection of data grouped together in memory. Structs are declared with the \textbf{struct} keyword.

\noindent \textit{Examples:}
\begin{minted}[xleftmargin=0.3cm,bgcolor=bg]{hlsl}
struct ps_input{
  float4 vPosition : SV_Position;
  float4 Color : COLOR;
  float3 Normal : NORMAL;
};

struct ps_output{
  float4 Color : SV_Target0;
};
\end{minted}

\noindent By using the structure operator ``\textbf{.}'' followed by the variable name, the data in a struct can be accessed.

\noindent \textit{Example:}
\begin{minted}[xleftmargin=0.3cm,bgcolor=bg]{hlsl}
ps_input Input;

float4 Color = Input.Color;
float3 Normal = Input.Normal;   
\end{minted}

\section{Semantics}
Semantics are Strings attached to the input and output variables of shaders\cite{HLSLSemantic}.
These Strings are used by the graphics pipeline to identify the meaning of the variable.
If there are multiple variables with the same semantic meaning, an Integer can be appended to the semantic.

\noindent \textit{Examples:}
\begin{minted}[xleftmargin=0.3cm,bgcolor=bg]{hlsl}
  //Position is treated as a Vertex Position.
  float4 Position : POSITION; 
  
  //vPosition is treated as a Pixel Location. 
  float2 vPosition : VPOS;    
  
  //Color is treated as a Color.
  float3 Color : COLOR1;      
\end{minted}

\section{Flow Control}
Since HLSL is a language for GPUs, and GPUs are made for Single Instruction Multiple Data (SIMD) parallel execution,
flow control is different to cpu flow control.
Branches on the cpu are simple, because comparisons operate on a single value.
Therefore the cpu is able to use the result to choose which branch to take.
Since GPUs are SIMD and operate on multiple values, branching can no longer be decided by a single value,
since different branches would require different instructions, which is does not work with SIMD.\@\cite{NvidiaFlowControl}
Loops have to wait, until all loop conditions are equal to 0.
HLSL uses c's flow control syntax, but adds additional options to it.

\noindent The keywords are:
\begin{itemize}
\item\textbf{if}
\item\textbf{switch}
\item\textbf{do}
\item\textbf{while}
\item\textbf{for}
\item\textbf{break}
\item\textbf{continue}
\item\textbf{discard}
\end{itemize}

\noindent The \textbf{discard} keyword is added by HLSL.\@ It can only be used in a pixel shader. 
If a discard statement is executed, the shader will not output the result of the pixel\cite{HLSLDiscard}.

\noindent \textit{Example:} 
\begin{minted}[xleftmargin=0.3cm,bgcolor=bg]{hlsl}
  //The shader will not output anything,
  //if the input's position has a positive z value.
  if(input.Position.z > 0){
    discard;
  }
\end{minted}

\noindent HLSL adds two attributes to \textbf{if}-statements:
\begin{itemize}
  \item \textbf{branch}: Executes only one side, depending on the condition.

  \item \textbf{flatten}: Executes the \textit{true} and \textit{false} side of the if-statement and decides, which one use, based on the condition. 
\end{itemize}


\section{Intrinsics}
Intrinsic functions are functions, that map directly to Assembly instructions.\\
Common examples are:
\begin{itemize}
\item\textbf{dot}: Calculates the dot product of two vectors
\item\textbf{mul}: Matrix multiplication
\item\textbf{cos}: Calculates the cosine
\end{itemize}

\section{Shader Models}
Shader Models specify, which features can be used in shaders using the Shader Model.
New Shader Models introduce new features or adjust existing features.
Each Shader Model specifies Shader Profiles\cite{HLSLShaderModel}, which are used when compiling the shaders.
Every Shader Type has its own shader profile, available in different versions.

\noindent \textit{Example:}

Shader Model 4 introduced the shader profiles cs\_4\_0 and cs\_4\_1. Both are shader profiles for compute shaders.

\section{Graphics Pipeline}
This section describes the Direct3D11 graphics pipeline, by looking at the state of each stage in the frame for this image and how each stage transforms its input.

\noindent \includegraphics[height= 0.3\textheight]{Assets/FinalImage.png}

\subsection{CPU Data}
The process starts at the cpu. The cpu has to decide what data the GPU has to transform and in which way.
The first step is to pass the Cube data to the gpu. This is done by creating a Vertex Buffer and an Index Buffer, 
that contain the necessary information for the input assembler.

\noindent \textbf{This is the data used:}

\begin{minted}[xleftmargin=0.3cm,bgcolor=bg,linenos]{hlsl}
float CubeVertices[]{
  /*Pos*/ -0.50f,-0.50f, 0.00f, /*COLOR*/ 0.00f, 0.00f, 0.00f, 1.00f,
  /*Pos*/  0.50f,-0.50f, 0.00f, /*COLOR*/ 0.00f, 0.00f, 1.00f, 1.00f,
  /*Pos*/ -0.50f, 0.50f, 0.00f, /*COLOR*/ 0.00f, 1.00f, 0.00f, 1.00f,
  /*Pos*/  0.50f, 0.50f, 0.00f, /*COLOR*/ 0.00f, 1.00f, 1.00f, 1.00f,
  /*Pos*/ -0.50f,-0.50f, 1.00f, /*COLOR*/ 1.00f, 0.00f, 0.00f, 1.00f,
  /*Pos*/  0.50f,-0.50f, 1.00f, /*COLOR*/ 1.00f, 0.00f, 1.00f, 1.00f,
  /*Pos*/ -0.50f, 0.50f, 1.00f, /*COLOR*/ 1.00f, 1.00f, 0.00f, 1.00f,
  /*Pos*/  0.50f, 0.50f, 1.00f, /*COLOR*/ 1.00f, 1.00f, 1.00f, 1.00f,
  };
UINT CubeIndices[]{
  //Front Face
  0,2,3,
  0,3,1,
  //Back Face
  5,6,4,
  5,7,6,
  //Left Face
  4,6,2,
  4,2,0,
  //Right Face
  1,3,7,
  1,7,5,
  //Top Face
  2,6,7,
  2,7,3,
  //Bottom Face
  0,4,5,
  0,5,1
};
\end{minted}
The data describes 8 vertices, each one with a different color.

\subsection{Input Assembler Stage}
The input assembler(IA) stage is responsible for transforming the vertices and indices into primitives, that can be processed.
The primitive topology can be set by the cpu. The tesslation stage is the only stage that requires a specific topology. It requires patchlist topologies.
To actually see  what the stages before the tesselation stage do to the vertices in the example, a trianglelist topology is used. 
Since the topology doesn't change the output of the stages before the tesselation stage, a different topology can be used to properly visualize the stages before.

\subsection{Required Pipeline Stages}
To see the an image from the data that has been passed to the cpu, a few pipeline stages must be active\cite{HLSLVS}.
In the cube example, vertex shader and pixel shader are active and only pass the data to the next stage without modifying it.
The rasterizer stage cannot be programmed by a shader, but its state can be changed by the cpu. If no rasterizer state is set, the default state is used.

\noindent This is the output after the IA stage and only passing through data.

\noindent \includegraphics[height= 0.2\textheight]{Assets/PassThrough.png}

\subsection{Rasterizer Stage}
This stage is the last stage before the pixel shader.
Because the primitves are described by vectors in the preceeding stages,
the rasterizer is responsible for transforming the data from the previous stages into a raster image. 
The rasterizer takes the primitives and calculates which pixels in the image are covered by a primitive.
There are different
\href{https://learn.microsoft.com/en-us/windows/win32/direct3d11/d3d10-graphics-programming-guide-rasterizer-stage-rules}
{rasterization rules}, depending on the primitve type, multisampling and anti-aliasing.

\noindent The rasterizer cannot be programmed with a shader like the other stages, but some settings can be set by the cpu.
It is possible to specify if the primitives are filled or a wireframe, enabling or disabling multisampling, enabling or disabling anti-aliasing.
More settings can be found in the \href{https://learn.microsoft.com/en-us/windows/win32/api/D3D11/ns-d3d11-d3d11_rasterizer_desc}{Direct3D reference}.

\noindent To properly see the effect of the different stages in the example, 
the rasterizer is the first stage that is changed, even though it would be further down the pipeline.
The rasterizer is set to wireframe mode.

\noindent \includegraphics[height= 0.2\textheight]{Assets/Rasterizer.png}

\noindent Notice, that the color seems to be dissapearing on the bottom left side. This is because the rasterizer is not only deciding, which pixels are filled,
it also does other things like interpolating between per-vertex data. The vertex on the bottom left is the same color as the background 
and the closer a pixel gets to this vertex, the more the pixels color is influenced by the vertex color.

\subsection{Vertex Shader Stage}
The vertex shader stage is the first stage after the input assembler stage. It is responsible for placing the primitves at the correct position on the screen.
This stage is where rotations, scaling and translations or per-vertex lighting happens. The vertex shader is executed for every vertex passed from the input assembler\cite{HLSLVS}, 
but operates only at one at a time. 

\noindent This is the stage, where the cube is being rotated and scaled down and the projected with an orthographic projection.

\noindent \includegraphics[height= 0.2\textheight]{Assets/VertexShader.png}

\href{https://gist.github.com/HatecrewJP/1a5f491b23e5101acf1bc5c783f65ece}{VertexShader}
\begin{minted}[xleftmargin=0.3cm,bgcolor=bg,linenos]{hlsl}
vs_output VSEntry(const vs_input input){
  vs_output output;

  float4 Input =  float4(input.vPosition,1);
  float4x4 OrthographicProjectionMatrix = {
  	1.0f,0.0f,0.0f,0.0f,
  	0.0f,1.0f,0.0f,0.0f,
  	0.0f,0.0f,-1.0f,0.0f,
  	0.0f,0.0f,0.0f,1.0f};
  float4 Scaling = {0.5f,0.5f,0.5f,1};

  Input *= Scaling;
  Input = RotationYaw(Input,RotationAngle);
  Input = RotationPitch(Input,RotationAngle);
  Input = RotationRoll(Input,0);
  Input.x /= (Width/Height);
  output.vPosition = mul(Input,OrthographicProjectionMatrix);
  output.Color =  input.Color;
  output.Normal = float3(0,0,0);
  return output;
}
\end{minted}

\noindent This is the main function of the vertex shader. Most of the code is simple, but there are two interesting lines, line 18 and 19 and the uniform declared variables.

\noindent Variables declared as \textbf{uniform} have the same value for all invokations within a render pass and are constant.\cite{HLSLVars}

\noindent The position data is normalized, in the intervall [-1,1], but the image it is rendered to is not square.
Without line 18, the cube would be streched along the x-axis. To compensate for that, the x coordinates must be divided by the aspect ratio (Width/Height).

\noindent Line 19 is interesting, because it is using the \textbf{mul} intrinsic. This intrinsic has to be used for efficient matrix multiplication.

\subsection{Tesselation Stages}
The tesselation stages enable hardware subdivision, which can be used to adjust the level of detail.
There are three stages, which must all be enabled concurrently: The hull shader stage, the tesselation stage and the domain shader stage.

\noindent The \textbf{Hull Shader Stage} requires the input topology to be a list of control points.
Now the are two phases done by the hull shader stage in parallel\cite{HLSLTess}: One phase is generating control-points for a patch from the control points, based on a specified geometry.
The other phase is generating additional data, patch-constant data, for the patch, which determines how the patch is subdivided.

\noindent Next is the \textbf{Tesselator Stage}, which cannot be programmed. It takes the patch generated by the hull shader and the patch-constant data as input 
and outputs the surface topology and UV(W)-coordinates to the domain stage\cite{HLSLTess}.

\noindent The \textbf{Domain Shader Stage} is responsible for calculating the positions of the new vertices based on the patch from the hull shader and the UV(W)-coordinates from the tesselator.

\href{https://gist.github.com/HatecrewJP/8898a51f45c2937ac1964ce53ce62eea}{Hull Shader} and 
\href{https://gist.github.com/HatecrewJP/97633439372d09aa21b89a5f4318541a}{Domain Shader}
\begin{minted}[xleftmargin=0.3cm,bgcolor=bg,linenos]{hlsl}
ConstantOutputType PatchConstantFunction(
  InputPatch<HsInput,3> inputPatch, 
  uint PatchID : SV_PrimitiveID)
{
  ConstantOutputType Output;

  Output.Edges[0] = 1;
  Output.Edges[1] = 1;
  Output.Edges[2] = 1;

  Output.inside = 2;

  return Output;
}


[domain("tri")]
[partitioning("pow2")]
[outputtopology("triangle_cw")]
[outputcontrolpoints(3)]
[patchconstantfunc("PatchConstantFunction")]
HsOutput HSEntry(
  InputPatch<HsInput,3>patch, 
  uint PointID : SV_OutputControlPointID, 
  uint PatchID : SV_PrimitiveID)
{
  HsOutput Output;
  Output.Color = patch[PointID].Color;
  Output.Position = patch[PointID].Position;

  return Output;
}





[domain("tri")]
DSOutput DSEntry(
  ConstantOutputType input, 
  float3 UVWCoord : SV_DomainLocation, 
  const OutputPatch<DSInput,3> patch)
{
  DSOutput Output;
  Output.Position = UVWCoord.x * patch[0].Position
                  + UVWCoord.y * patch[1].Position 
                  + UVWCoord.z * patch[2].Position;
  Output.Color = (patch[0].Color + patch[1].Color + patch[2].Color)/3.0f;
  Output.Color.w = 1.0f;
  Output.Normal = float3(0,0,0);
  return Output;
}
\end{minted}
 
\noindent The hull shader provides two functions. The patch-constant function, which generates the patch-constant data, 
and the main function, which generates the control-points for the patch.

\noindent The main function of the domain shader calculates the new vertices.

\noindent Before each main function properties must be declared within square brackets. The hull shader stage needs to know the control patch topology,
the target topology, how many control points can be generated by the hull shader 
and which function is used to calculate the patch-constant data.
The domain shader needs to know the control patch topology.

\noindent This is the cube, after tesselation:

\noindent \includegraphics[height= 0.2\textheight]{Assets/TesselationStage.png}


\subsection{Geometry Shader Stage}

Geometry Shaders are the last stage before the rasterizer and can be used to generate new primitives. Unlike the previous stages, which operate on single vertices, geometry shaders
are invoked for each primitive coming from the previous stage\cite{HLSLGS}. 
The data from the input primitives can now be used to generate new vertices, which define the new primitives.
The amount of generated vertices is not fixed, but a static upper limit must be defined.

\noindent \includegraphics[height= 0.2\textheight]{Assets/GeometryShader.png}

\href{https://gist.github.com/HatecrewJP/ac2ab08758fb5adff49a303aa3664ee2}{Geometry Shader}
\begin{minted}[xleftmargin=0.3cm,bgcolor=bg,linenos]{hlsl}
[maxvertexcount(9)]
void GSEntry(
  triangle GSInput InputTri[3] : SV_Position, 
  inout TriangleStream<GSOutput> OutStream)
{
  float4 MidPoint = (InputTri[0].Position + InputTri[1].Position 
    + InputTri[2].Position)/3;
  MidPoint.w = 1.0f;
  float4 MidPointColor = float4((InputTri[0].Color.xyz 
    + InputTri[1].Color.xyz + InputTri[2].Color.xyz)/3,1);
  GSOutput Output;
  Output.Color =  MidPointColor;
  Output.Color.w = 1.0f;

  Output.Normal = CalculateNormalFromTriangle(InputTri[0].Position,
                                              InputTri[1].Position,MidPoint);
  Output.Color.xyz = (InputTri[0].Color.xyz + InputTri[1].Color.xyz 
    + MidPointColor.xyz)/3.0f;
  Output.Pos = InputTri[0].Position;
  OutStream.Append(Output);
  Output.Pos = InputTri[1].Position;
  OutStream.Append(Output);
  Output.Pos = MidPoint;
  OutStream.Append(Output);

  Output.Normal = CalculateNormalFromTriangle(InputTri[1].Position,
                                              InputTri[2].Position,MidPoint);
  Output.Color.xyz = (InputTri[1].Color.xyz + InputTri[2].Color.xyz 
    + MidPointColor.xyz)/3.0f;
  Output.Pos = InputTri[2].Position;
  OutStream.Append(Output);

  Output.Normal = CalculateNormalFromTriangle(InputTri[2].Position,
                                              InputTri[0].Position,MidPoint);
  Output.Color.xyz = (InputTri[2].Color.xyz + InputTri[0].Color.xyz 
    + MidPointColor.xyz)/3.0f;
  Output.Pos = InputTri[0].Position;
  OutStream.Append(Output);
  OutStream.RestartStrip();
}
\end{minted}



\noindent The Geometry Shader subdivides the cube even further.
Each time the geometry shader is executed, the input primitves are used to calculate the triangles midpoint. This midpoint is then used to generate
three triangles, each made by the midpoint and two vertices from the input, 
which are then appended to the output stream as a \href{https://learn.microsoft.com/en-us/windows/win32/direct3d9/triangle-strips}{triangle strip}.

\noindent The geometry shader output is passed down to the rasterizer stage.

\subsection{Pixel Shader Stage}
After the rasterizer generated an image from the primitives, the pixel shader is invoked per-pixel and 
combines textures, buffer data and per-vertex values\cite{HLSLPS} to calculate color- and optional depth-data of the pixel in the image.

\noindent \includegraphics[height= 0.2\textheight]{Assets/PixelShader.png}

\href{https://gist.github.com/HatecrewJP/c22c3aa9650dfd28dac0e9172eba6dcd}{Pixel Shader}
\begin{minted}[xleftmargin=0.3cm,bgcolor=bg,linenos]{hlsl}
#define DEBUGZ 0

cbuffer CBuffer : register(b0){
  float RotationAngle : packoffset(c0.x);
  float Width  : packoffset(c0.y);
  float Height : packoffset(c0.z);
  float ColorR : packoffset(c1.x);
  float ColorG : packoffset(c1.y);
  float ColorB : packoffset(c2.x);
  float ColorA : packoffset(c2.y);
};

ps_output PSEntry(const ps_input input){
  ps_output output;
  output.Color = input.Color;
  [branch]if(input.Color.r == 1.0f){
  	output.Color = float4(ColorR,ColorG,ColorB,ColorA);
  }


  [flatten]if(output.Color.r == 0.0f){
  	output.Color = float4(ColorR,ColorG,ColorB,ColorA);
  }

#if DEBUGZ
  if(input.vPosition.z > 0){
  	output.Color = float4(1,0,0,1);
  }
#endif
  return output;
}
\end{minted}

\noindent The pixel shader animates all pixels, where the red channel is equal to 1 or 0;
The color of the animation comes from a constant buffer. The cpu is responsible for updating the color. 

\noindent Notice that the variables in the constant buffer can be used like global variables.


\subsection{Output Merger Stage}

The last graphics pipeline stage is the output merger. Its job is to generate the final pixel color\cite{HLSLOMS}, 
by blending the color output of the pixel shader with the depth data. If no depth data is specified, the final pixel color
is the color coming from the pixel shader.


\subsection{Compute Shader}

Direct3D11 introduced computer shaders. These shaders are general purpose shaders, which are no part of the graphics pipeline.
Instead it is possible abuse the parallel design of the gpu to perform computations on the gpu, which aren't necessarily graphics related.

\noindent Even though compute shaders are independent from the graphics pipeline, both can be used together.

\noindent \includegraphics[height= 0.2\textheight]{Assets/ComputeShader.png}

\noindent Compute shaders can be used for post-processing effects. 
The example uses a compute shader that scales the red channel down for every pixel in the image, excluding all pixels with an rgba value of 0xffffffff.

\href{https://gist.github.com/HatecrewJP/90dc57625a3fe2f78e667d3e2613297a}{ComputerShader}
\begin{minted}[xleftmargin=0.3cm,bgcolor=bg,linenos]{hlsl}
RWTexture2D <unorm float4> SwapChainImageCopy : register(u0);

[numthreads(1,1,1)]
void CSEntry(uint3 GroupID: SV_GroupID){
  uint2 Coordinate = uint2(GroupID.x,GroupID.y);

  float4 Color = SwapChainImageCopy[Coordinate];
  bool ColorIsWhite = (Color.x+Color.y+Color.z+Color.w) == 4;

  [flatten]if(!ColorIsWhite){
  	Color.r *= 0.7f;
  }
  SwapChainImageCopy[Coordinate] = Color;
}
\end{minted}


\noindent Compute shaders need to specify how many threads 
are spawned when the shader is invoked. This is done by providing three values, x, y and z, in the \textbf{numthreads} property\cite{HLSLCreateCS}.

\subsection{Final Image}
After all these steps the image can be rendered on the screen.

\includegraphics[height= 0.2\textheight]{Assets/FinalImage.png}

\section{Concepts without explanation}
\begin{itemize}
  \item GPU Registers
  \item SIMD
  \item primitives
  \item multisampling
  \item anti-aliasing
  \item interpolation
  \item per-vertex data
  \item translation
  \item per-vertex lighting
  \item render pass
  \item normalized data
  \item Aspect Ratio
  \item UV-coordinates
  \item domain
  \item control-points
  \item subdivision
  \item Primtive strips
  \item Depth Buffering
  \item threads
\end{itemize}


\bibliographystyle{ACM-Reference-Format}
\bibliography{Sources}
\end{document}




