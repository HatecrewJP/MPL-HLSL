\documentclass[acmsmall,screen,nonacm,review]{acmart}
\usepackage{minted}
\usepackage{hyperref}

\title{An Introduction to HLSL}

\begin{document}
\maketitle

\section{Introduction}

\section{HLSL Language}
The HLSL Language is similar to the c programming language.
A lot of it has been taken over from c
but some new things, that help programming the graphics pipeline, have been added.
This section describes the language syntax.


\subsection{Variable Declaration Syntax}
This section describes how variables are declared. 

Some options are specific to more advanced features that are either explained later or are not discussed in this report.

\small[Storage\_Class] [Type\_Modifier] Type Name[Index] [: Semantic] [: Packoffset] [: Register]; [Annotations] [= Initial\_Value] 
\normalsize

Storage\_Class:

Parameter that gives the compiler a hint about scope and lifetime.

Examples:

static: The variable is initialized once and persist after leaving the scope.

uniform: The variable's data remains the same throughout the execution of the shader.


Type\_Modifier:

Optional modifier.

Example:

const: The variable cannot be changed.

Type:
The type of the data being stored in the variable. It can be any of the HLSL Data Types.

Examples:
\textbf{int}: signed 32-bit integer
\textbf{dword}: unsigned 32-bit integer  
\textbf{double}: 64-bit floating point number

Name:

An ASCII String that uniquely identifies the variable.

Index:

Providing a positive Integer as the Index makes the variable an Array.

Examples:
\begin{minted}{hlsl}
  float MyArray1[3] //An Array of 3 floats.
  uint MyArray2[32] //An Array of 32 uints. 
\end{minted}


Semantic:

A String attached to shader inputs and outputs. 

Packoffset:

Each register has 4 32-bit components. When passing variables in a buffer to the GPU, some variables may not use all 4 subcomponents.
If each variable uses only one of those components and has its own register, a lot of memory would be wasted.
To avoid wasting memory, data can be packed into registers, if the variables components are small enough to fully fit into the register.

To retrieve packed data and store it into a variable, packoffset needs the buffer slot and component slot, from which the data should be used.

Examples:

Valid:

\begin{minted}{hlsl}
  cbuffer CBuffer{
	float RotationAngle : packoffset(c0.x);
	float Width         : packoffset(c0.y);
	float Height        : packoffset(c0.z);
	float ColorR        : packoffset(c1.x);
	float ColorG        : packoffset(c1.y);
	float ColorB        : packoffset(c1.z);
	float ColorA        : packoffset(c1.w);
};
\end{minted}

Invalid:
\begin{minted}{hlsl}
  cbuffer{
      float2 x: packoffset(c0);   // x uses 2 components 
      float3 y: packoffset(c0.z); // y uses 3 components
  }
\end{minted}

Register:

Assigns one or more registers to a variable. The register can be specified by the type, the buffer slot index and a subcomponent.

\begin{minted}{hlsl}
  float RotationAngle : register(b0[0]); //Constant Buffer at Index 0, Data at offset 0
  float Width         : register(b0[1]); //Constant Buffer at Index 0, Data at offset 1
  float Height        : register(b0[2]); //Constant Buffer at Index 0, Data at offset 2
\end{minted}

Annotations:

Additional information added to a variable. Annotations are used by Effects.

Examples from the HLSL reference:
\begin{minted}{hlsl}
  int j <int bambam=30; string blacksheep="Goodbye There";> = 5 ;
  float y <float y=2.3;> = 2.3, z <float y=1.3;> = 1.3 ;
  half w <half GlobalW = 3.62;>;
\end{minted}

Initial\_Value:

The Value, that the variable is initialized with.

\subsection{Data Types}
HLSL supports various data types. This section gives an overview over the different Data Types.

\subsubsection{Scalar}
Scalar variables use a single component of a register. There are different kinds of scalar data types.

The main ones are:

-\textbf{bool}: true or false

-\textbf{int}: 32-bit signed integer

-\textbf{uint}: 32-bit unsigned integer

-\textbf{half}: 16-bit floating point

-\textbf{float}: 32-bit floating point

-\textbf{double}: 64-bit floating point

-\textbf{snorm float}: 32-bit signed-normalized float in the intervall[-1,1]

-\textbf{unorm float}: 32-bit unsigned-normalized float in the intervall[0,1]

Examples:

\begin{minted}{hlsl}
  float RotationAngle;
  bool  RedChannelActive;
\end{minted}

\subsubsection{Vector}

Vector's are variables that use up to 4 components. The type can be any of the scalar data types. A vector can be declared in two ways.

Declaration with a template:

vector<Type, Components = [1,4]>

Example:
\begin{minted}{hlsl}
   // A four component float

  vector<float, 1> RotationAngle; // A single component float, the same as a scalar float
\end{minted}

Type declaration:

HLSL provides predefined macros for vectors of built-in scalar types.
This simplifies the vector declaration.

Macro example:
\begin{minted}{hlsl}
  #define float4 vector<float,4> //predefined macro
  float4 Var1;
  //expands to
  vector<float,4> Var1;
\end{minted}


Examples:
\begin{minted}{hlsl}
  float4 vPosition :SV_Position;
	float4 Color : COLOR;
	float3 Normal : NORMAL;
\end{minted}

Accessing a component can be done by using the structure operator \textbf{.} followd by the letters xyzw or rgba.

Examples:
\begin{minted}{hlsl}
  Vector2.xy  //acceses the first and second component
  Vector3.rgb //acceses the first second and third component
  Vector4.yzw //accesses the second, third and fourth component
\end{minted}

The letters \textit{xyzw} or \textit{rgba} correspond to the components of a Vector.
\textit{x} and \textit{r} correspond to the first component.
\textit{y} and \textit{g} correspond to the second component.
\textit{z} and \textit{b} correspond to the third component.
\textit{w} and \textit{a} correspond to the fourth component.



\subsubsection{Matrix}
Matrices are two dimensional grids made of scalar components. HLSL allows matrices with one to four columns and rows respectively.

Just like Vector's, there are two ways of declaring matrices.

Template Declaration:

matrix<Type, Rows = [1,4], Cols = [1,4]>

Type Declaration:

There are also macros for matrices provided by HLSL:

\begin{minted}{hlsl}
  #define float3x2 matrix<float,3,2>  //predefined macro
  float3x2 Mat3x2;
  //expands to
  matrix<float,3,2> Mat3x2;

\end{minted}

Examples:
\begin{minted}{hlsl}
  float4x4 RollRotationMatrix = {         // A 4 by 4 Matrix
	1.0f,0.0f,0.0f,0.0f,
	0.0f,cos(RadAngle),-sin(RadAngle),0.0f,
	0.0f,sin(RadAngle),cos(RadAngle),0.0f,
	0.0f,0.0f,0.0f,1.0f};
  
  int3x2 Matrix = {}; // A 3 by 2 matrix, initialized with 0
\end{minted}

Accessing a component of a matrix can be done multiple ways using the structure operator \textbf{.}.

One way is to follow \textbf{.} by \_mRC, where R and C correspond to Row/Column indices.

Example: 

\begin{minted}{hlsl}
  //Accesses the first row of the RollRotationMatrix
  float4 FirstRow = RollRotationMatrix._m00_m01_m02_m03; 
\end{minted}

\subsubsection{Buffers}

Each stage of the graphics pipeline has the possibilty to hold references to buffers provided by the cpu.

To acces these buffers, HLSL provides multiple buffer types.

Examples of Buffer types:

\textbf{cbuffer}:  A Constant Buffer Object

\textbf{RWBuffer}:  A Read/write Buffer Object

\textbf{RWTexture2D}: A 2d Texture Buffer Object

Buffer Examples:

\begin{minted}{hlsl}
  // A Read/Write Texture2D Buffer Object 
  RWTexture2D <float4> TextureCopy;
  
  //A Constant Buffer Object
  cbuffer CBuffer{                          
    float RotationAngle : packoffset(c0.x);
    float Width  : packoffset(c0.y);
    float Height : packoffset(c0.z);
    float ColorR : packoffset(c1.x);
    float ColorG : packoffset(c1.y);
    float ColorB : packoffset(c2.x);
    float ColorA : packoffset(c2.y);
  };
\end{minted}

Each buffer object has its own way of accessing the data.

Examples:

Variables in a constant buffer can be used as global variables.
\begin{minted}{hlsl}
  //Accessing the CBuffer Object from the previous example.
  RotationAngle +=1;
\end{minted}

RWTexture2D data can be accesed like an array.
\begin{minted}{hlsl}
  //Accessing the TextureCopy from the previous example
  float4 Color = TextureCopy[0];
\end{minted}

\subsubsection{Struct}
A Struct is a collection of data grouped together in memory.

Structs are declared with the \textbf{struct} keyword.

Examples:

\begin{minted}{hlsl}
  struct ps_input{
    float4 vPosition : SV_Position;
    float4 Color : COLOR;
    float3 Normal : NORMAL;
  };

  struct ps_output{
    float4 Color : SV_Target0;
  };
\end{minted}

By using the structure operator \textbf{.} followed by the variable name, the data in a struct can be accessed.

Example:

\begin{minted}{hlsl}
  ps_input Input;

  float4 Color = Input.Color;
  float3 Normal = Input.Normal;
   
\end{minted}

\subsection{Semantics}
Semantics are Strings attached to the input and output variables of shaders.
These Strings are used by the graphics pipeline to identify the meaning of the variable.
If there are multiple variables with the same semantic meaning, an Integer can be appended to the semantic.

Examples:
\begin{minted}{hlsl}
  //Position is treated as a Vertex Position.
  float4 Position : POSITION; 
  
  //vPosition is treated as a Pixel Location. 
  float2 vPosition : VPOS;    
  
  //Color is treated as a Color.
  float3 Color : COLOR1;      
\end{minted}

\subsection{Flow Control}
Since HLSL is a language for GPU's, and GPU's are made for Single Instruction Multiple Data (SIMD) paralell execution,
flow control is different to cpu flow control.

Branches on the cpu are simple, because comparisons operate on a single value.
Therefore the cpu is able to use the result to choose which branch to take.
Since GPU's are SIMD and operate on multiple values, branching can no longer be decided by a single value,
since different branches would require different instructions, which is does not work with SIMD.

Loops have to wait, until all loop conditions are equal to 0.

HLSL uses c's flow control syntax, but adds additional options to it.

The keywords are:
\textbf{if}
\textbf{switch}
\textbf{do}
\textbf{while}
\textbf{for}
\textbf{break}
\textbf{continue}
\textbf{discard}

The \textbf{discard} keyword is added by HLSL. It can only be used in a pixel shader. 

If a discard statement is executed, the shader will not output the result of the pixel.

Example: 
\begin{minted}{hlsl}
  //The shader will not output anything,
  //if the input's position has a positive z value.
  if(input.Position.z > 0){
    discard;
  }
\end{minted}

HLSL adds two attributes to \textbf{if}-statements:

\textbf{branch}:
Executes only one side, depending on the condition.

\textbf{flatten}: 
Executes the \textit{true} and \textit{false} side of the if-statement and decides, which one use, based on the condition. 

\subsection{Intrinsics}
Intrinsic functions are functions, that map directly to Assembly instructions.

Common examples are:
\textbf{dot}: Calculates the dot product of two vectors
\textbf{mul}: Matrix multiplication
\textbf{cos}: Calculates the cosine

\subsection{Shader Models}
Shader Models specify, which features can be used in shaders using the Shader Model.
New Shader Models introduce new features or adjust existing features.
Each Shader Model specifies Shader Profiles, which are used when compiling the shaders.
Every Shader Type has its own shader profile, available in different versions.

Example:

Shader Model 4 introduced the shader profiles cs\_4\_0 and cs\_4\_1. Both are shader profiles for compute shaders.

\section{Graphics Pipeline}
This section describes the Direct3D11 graphics pipeline, by looking at the state of each stage in the frame for this image and how each stage transforms its input.

\includegraphics[height= 0.3\textheight]{Assets/FinalImage.png}

\subsection{CPU Data}
The process starts at the cpu. The cpu has to decide what data the GPU has to transform and in which way.

The first step is to pass the Cube data to the gpu. This is done by creating a Vertex Buffer and an Index Buffer, 
that contain the nescessary information for the input assembler.

This is the data used:

\begin{minted}{hlsl}
  float CubeVertices[]{
    /*Pos*/ -0.50f,-0.50f, 0.00f, /*COLOR*/ 0.00f, 0.00f, 0.00f, 1.00f,
    /*Pos*/  0.50f,-0.50f, 0.00f, /*COLOR*/ 0.00f, 0.00f, 1.00f, 1.00f,
    /*Pos*/ -0.50f, 0.50f, 0.00f, /*COLOR*/ 0.00f, 1.00f, 0.00f, 1.00f,
    /*Pos*/  0.50f, 0.50f, 0.00f, /*COLOR*/ 0.00f, 1.00f, 1.00f, 1.00f,
    /*Pos*/ -0.50f,-0.50f, 1.00f, /*COLOR*/ 1.00f, 0.00f, 0.00f, 1.00f,
    /*Pos*/  0.50f,-0.50f, 1.00f, /*COLOR*/ 1.00f, 0.00f, 1.00f, 1.00f,
    /*Pos*/ -0.50f, 0.50f, 1.00f, /*COLOR*/ 1.00f, 1.00f, 0.00f, 1.00f,
    /*Pos*/  0.50f, 0.50f, 1.00f, /*COLOR*/ 1.00f, 1.00f, 1.00f, 1.00f,
    };
  UINT CubeIndices[]{
    //Front Face
    0,2,3,
    0,3,1,
    //Back Face
    5,6,4,
    5,7,6,
    //Left Face
    4,6,2,
    4,2,0,
    //Right Face
    1,3,7,
    1,7,5,
    //Top Face
    2,6,7,
    2,7,3,
    //Bottom Face
    0,4,5,
    0,5,1
  };
\end{minted}

The data describes 8 vertices, each one with a different color.

\subsection{Input Assembler Stage}
The input assembler(IA) stage is responsible for transforming the vertices and indices into primitives, that can be processed.
The primitive topology can be set by the cpu. The tesslation stage operates operates on patchlist topologies.
To actually see what the stages before do to the vertices, a trianglelist topology is used. 
Since the topology doesn't change the output of the stages before the tesselation stage, a different topology can be used to properly visualize the stages before.


\subsection{Required Pipeline Stages}
To see the an image from the data that has been passed to the cpu, a few pipeline stages must be active.
In the cube example, vertex shader and pixel shader are active and only pass the data to the next stage without modifying it.
The rasterizer stage cannot be programmed by a shader, but its state can be changed by the cpu. If no rasterizer state is set, the default state is used.

This is the output after the IA stage and only passing through data.

\includegraphics[height= 0.2\textheight]{Assets/PassThrough.png}

\subsection{Rasterizer Stage}
This stage is the stage before the pixel shader.
Because the primitves are described by vectors in the preceeding stages,
the rasterizer is responsible for transforming the data from the previous stages into a raser image. 
The rasterizer takes the primitives and calculates which pixels in the image are covered by a primitive.
There are different
\href{https://learn.microsoft.com/en-us/windows/win32/direct3d11/d3d10-graphics-programming-guide-rasterizer-stage-rules}
{rasterization rules}, depending on the primitve type, multisampling and anti-aliasing.

The rasterizer cannot be programmed with a shader like the other stages, but some settings can be set by the cpu.
It is possible to specify if the primitives are filled or a wireframe, enabling or disabling multisampling, enabling or disabling anti-aliasing.
More settings can be found here: \url{https://learn.microsoft.com/en-us/windows/win32/api/D3D11/ns-d3d11-d3d11_rasterizer_desc}

To properly see the effect of the different stages in the example, 
the rasterizer is the first stage that is changed, even though it would be further down the pipeline.
The rasterizer is set to wireframe mode.

\includegraphics[height= 0.2\textheight]{Assets/Rasterizer.png}

Notice, that the color seems to be dissapearing on the bottom left side. This is because the rasterizer is not only deciding, which pixels are filled,
it also does other things like interpolating between per-vertex data. The vertex on the bottom left is the same color as the background 
and the closer a pixel gets to this vertex, the more the pixels color is influenced by the vertex color.

\subsection{Vertex Shader Stage}
The vertex shader stage is the first stage after the input assembler stage. It is responsible for placing the primitves at the correct position on the screen.
This stage is where rotations, scaling and translations or per-vertex lighting happens. The vertex shader is executed for every vertex passed from the input assembler, 
but operates only at one at a time. 

This is the stage, where the cube is being rotated and scaled down and the projected with an orthographic projection.

\includegraphics[height= 0.2\textheight]{Assets/VertexShader.png}

This is the main function of the \href{https://gist.github.com/HatecrewJP/1a5f491b23e5101acf1bc5c783f65ece}{vertex shader}.
\begin{minted}[linenos]{hlsl}
  vs_output VSEntry(const vs_input input){
    vs_output output;
    float4 Input =  float4(input.vPosition,1);

    float4x4 OrthographicProjectionMatrix = {
      1.0f,0.0f,0.0f,0.0f,
      0.0f,1.0f,0.0f,0.0f,
      0.0f,0.0f,-1.0f,0.0f,
      0.0f,0.0f,0.0f,1.0f};

    float4 Scaling = {0.5f,0.5f,0.5f,1};


    Input *= Scaling;
    //Rotation around the z-axis
    Input = RotationYaw(Input,RotationAngle);
    //Rotation around the y-axis
    Input = RotationPitch(Input,RotationAngle);
    //Rotation around the x-axis
    Input = RotationRoll(Input,0);
    Input.x /= (Width/Height);
    output.vPosition = mul(Input,OrthographicProjectionMatrix);
    output.Color =  input.Color;
    output.Normal = float3(0,0,0);
    return output;
  }
\end{minted}

Most of the code is simple, but there are two interesting lines, line 18 and 19.

The position data is normalized, in the intervall [-1,1], but the image it is rendered to is not square.
Without line 18, the cube would be streched along the x-axis. To compensate for that, the x coordinates must be divided by the aspect ratio (Width/Height).

Line 19 is interesting, because it is using the \textbf{mul} intrinsic. This intrinsic has to be used for efficient matrix multiplication.

\subsection{Tesselation Stages}

The tesselation stages enable hardware subdivision, which can be used to adjust the level of detail.
There are three stages, which must all be enabled concurrently: The hull shader stages, the tesselation stage and the domain shader stage.

The \textbf{Hull Shader Stage} requires the input topology to be a list of control points.
Now the are two phases done by the hull shader stage in parallel: One phase is generating control-points for a patch from the control points, based on a specified geometry.
The other phase is generating additional data, patch-constant data, for the patch, which determines how the patch is subdivided.

Next is the \textbf{Tesselator Stage}, which cannot be programmed. It takes the patch generated by the hull shader and the patch-constant data as input 
and outputs the surface topology and UV(W)-coordinates to the domain stage.

The \textbf{Domain Shader Stage} is responsible for calculating the positions of the new vertices based on the patch from the hull shader and the UV(W)-coordinates from the tesselator.

\begin{minted}{hlsl}
    ConstantOutputType PatchConstantFunction(InputPatch<HsInput,3> inputPatch, uint PatchID : SV_PrimitiveID){
    ConstantOutputType Output;

    Output.Edges[0] = 1;
    Output.Edges[1] = 1;
    Output.Edges[2] = 1;

    Output.inside = 2;

    return Output;
  }
  
  [domain("tri")]
  [partitioning("pow2")]
  [outputtopology("triangle_cw")]
  [outputcontrolpoints(3)]
  [patchconstantfunc("PatchConstantFunction")]
  
  HsOutput HSEntry(InputPatch<HsInput,3>patch, uint PointID : SV_OutputControlPointID, uint PatchID : SV_PrimitiveID){
    HsOutput Output;
    Output.Color = patch[PointID].Color;
    Output.Position = patch[PointID].Position;

    return Output;
  }
  
  [domain("tri")]
  DSOutput DSEntry(ConstantOutputType input, float3 UVWCoord : SV_DomainLocation, const OutputPatch<DSInput,3> patch){
    DSOutput Output;
    Output.Position = UVWCoord.x * patch[0].Position + UVWCoord.y * patch[1].Position + UVWCoord.z * patch[2].Position;
    Output.Color = (patch[0].Color + patch[1].Color + patch[2].Color)/3.0f;
    Output.Color.w = 1.0f;
    Output.Normal = float3(0,0,0);
    return Output;

\end{minted}
 
The hull shader provides two functions. The patch-constant function, which generates the patch-constant data, 
and the main function, which generates the control-points for the patch.

The main function of the domain shader calculates the new vertices.

Before each main function properties must be declared within square brackets. The hull shader stage needs to know the control patch topology,
the target topology, how many control points can be generated by the hull shader 
and which function is used to calculate the patch-constant data.

The domain shader needs to know the control patch topology.

This is the cube, after tesselation:
\includegraphics[height= 0.2\textheight]{Assets/TesselationStage.png}


\subsection{Geometry Shader Stage}

\includegraphics[height= 0.2\textheight]{Assets/GeometryShader.png}

\subsection{Pixel Shader Stage}

\includegraphics[height= 0.2\textheight]{Assets/PixelShader.png}

\subsection{Output Merger Stage}

\subsection{Compute Shader}

\includegraphics[height= 0.2\textheight]{Assets/ComputeShader.png}

\subsection{FinalImage}

\includegraphics[height= 0.2\textheight]{Assets/FinalImage.png}

\section{Passing Data from the CPU to GPU}
\subsection{Constant Buffer}
\subsection{Unordered Access View}

\end{document}